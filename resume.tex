%-------------------------
% Resume in LateX
% Author : Sourabh Bajaj
% License : MIT
%------------------------

\documentclass[letterpaper,11pt]{article}

\usepackage{latexsym}
\usepackage[empty]{fullpage}
\usepackage{titlesec}
\usepackage{marvosym}
\usepackage[usenames,dvipsnames]{color}
\usepackage{verbatim}
\usepackage{enumitem}
\usepackage[hidelinks]{hyperref}
\usepackage{fancyhdr}
\usepackage[english]{babel}
\usepackage{tabularx}
\usepackage{kotex}
\input{glyphtounicode}

\pagestyle{fancy}
\fancyhf{} % Clear all header and footer fields
\fancyfoot{}
\renewcommand{\headrulewidth}{0pt}
\renewcommand{\footrulewidth}{0pt}

% Adjust margins
\addtolength{\oddsidemargin}{-0.5in}
\addtolength{\evensidemargin}{-0.5in}
\addtolength{\textwidth}{1in}
\addtolength{\topmargin}{-.5in}
\addtolength{\textheight}{1.0in}

\urlstyle{same}

\raggedbottom
\raggedright
\setlength{\tabcolsep}{0in}

% Sections formatting
\titleformat{\section}{
  \vspace{-4pt}\scshape\raggedright\large
}{}{0em}{}[\color{black}\titlerule \vspace{-5pt}]

% Ensure that generate PDF is machine readable/ATS parsable
\pdfgentounicode=1

%-------------------------
% Custom commands
\newcommand{\resumeItem}[2]{
  \item\small{
    \textbf{#1}{: #2 \vspace{-2pt}}
  }
}

% Just in case someone needs a heading that does not need to be in a list
\newcommand{\resumeHeading}[4]{
    \begin{tabular*}{0.99\textwidth}[t]{l@{\extracolsep{\fill}}r}
      \textbf{#1} & #2 \\
      \textit{\small#3} & \textit{\small #4} \\
    \end{tabular*}\vspace{-5pt}
}

\newcommand{\resumeSubheading}[4]{
  \vspace{-1pt}\item
    \begin{tabular*}{0.97\textwidth}[t]{l@{\extracolsep{\fill}}r}
      \textbf{#1} & #2 \\
      \textit{\small#3} & \textit{\small #4} \\
    \end{tabular*}\vspace{-5pt}
}

\newcommand{\resumeSubSubheading}[2]{
    \begin{tabular*}{0.97\textwidth}{l@{\extracolsep{\fill}}r}
      \textit{\small#1} & \textit{\small #2} \\
    \end{tabular*}\vspace{-5pt}
}

\newcommand{\resumeSubItem}[2]{\resumeItem{#1}{#2}\vspace{-4pt}}

\renewcommand{\labelitemii}{$\circ$}

\newcommand{\resumeSubHeadingListStart}{\begin{itemize}[leftmargin=*]}
\newcommand{\resumeSubHeadingListEnd}{\end{itemize}}
\newcommand{\resumeItemListStart}{\begin{itemize}}
\newcommand{\resumeItemListEnd}{\end{itemize}\vspace{-5pt}}

%-------------------------------------------
%%%%%%  CV STARTS HERE  %%%%%%%%%%%%%%%%%%%%%%%%%%%%


\begin{document}

%----------HEADING-----------------
\begin{tabular*}{\textwidth}{l@{\extracolsep{\fill}}r}
  \textbf{\Large 이재준} & Email: \href{mailto:jx2lee@dev.jaejun.lee.1991@gmail.com}{dev.jaejun.lee.1991@gmail.com} \\
  \href{https://www.jx2lee.kr}{jx2lee.kr} \\
\end{tabular*}


%-----------EDUCATION-----------------
\section{Education}
  \resumeSubHeadingListStart
    \resumeSubheading
      {국민대학교}{성북구, 대한민국}
      {자연과학대학 수학과; GPA: 3.42}{Mar 2010 -- Feb 2016}
  \resumeSubHeadingListEnd


%-----------EXPERIENCE-----------------
\section{Experience}
  \resumeSubHeadingListStart
    \resumeSubheading
      {\href{https://www.jx2lee.kr/career/bithumb}{Bithumb}}{강남구, 대한민국}
      {Software Engineer, Data}{Jan 2025 -- Present}
      \resumeItemListStart
        \resumeItem{High-volume, Optimized Data Platform}
          {체험할 수 없을 정도의 데이터를 다루며 효율적으로 수집/처리하는 업무에 집중했습니다. Lakehouse architecture 을 활용한 데이터 플랫폼을 개발 운영하며, 대용량 처리와 품질 향상에 중점을 두었어요}
        \resumeItem{Data API}
          {서비스 DB에서 실행하기 부담스러운 집계를 DW 에 위임, 이를 제공하는 API 를 개발했습니다. 비즈니스 요구사항을 분리하기 위해 Reverse ETL 을 이용했고, 모노레포로 구성해 빠른 개발 및 안전한 배포를 지향했어요.}
        \resumeItem{Data Outcome Measures}
          {그동안 경험한 데이터팀은 "도와주는" 역할의 이미지가 강했지만, 저는 "함께하는" 데이터팀을 꿈꿨습니다. 데이터가 어떻게 활용되고 있는지, 그 결과가 비즈니스에 어떤 영향을 미쳤는지 측정하는 지표를 개발하고 운영했어요.}
      \resumeItemListEnd
      
% --------Multiple Positions Heading------------
  %  \resumeSubSubheading
  %   {Software Engineer I}{Oct 2014 -- Sep 2016}
  %   \resumeItemListStart
  %      \resumeItem{Apache Beam}
  %        {Apache Beam is a unified model for defining both batch and streaming data-parallel processing pipelines}
  %   \resumeItemListEnd

%-------------------------------------------

    \resumeSubheading
      {\href{https://www.jx2lee.kr/career/coinone}{Coinone}}{영등포구, 대한민국}
      {Data Engineer}{Sep 2022 -- Jan 2025}
      \resumeItemListStart
        \resumeItem{Data Warehousing}
          {거래소 서비스에서 발생한 데이터를 클라우드 데이터웨어하우스에 적재하는 파이프라인을 개발하고 운영했습니다. 테이블 성격에 맞는 수집방식을 고민하고 신뢰할 수 있는 데이터를 제공하는 Data as a Product 에 집중했어요.}
        \resumeItem{Metric Store}
          {임직원들에게 다양하고 유용한 지표를 제공하고자 지표저장소를 만들고 운영했습니다. 그들이 보고싶어하는 지표(협의된)를 쉽게 찍어내는 데이터 프로덕트(show metric using cli)도 만들었습니다. 상상속에서 뛰놀던 지표를 만들어볼 수 있는 기회를 제공했어요.}
        \resumeItem{Self-Serve-Data}
          {원하는 데이터를 데이터 조직에 의존하지 않고 스스로 찾아 볼 수 있도록 환경을 제공했습니다. 도구(BI/디스커버리/워크플로우 오케스트레이샨)들을 컨테이너 환경에 맞게 잘 패킹하고, 단일 진실 공급원을 유지했어요.}
      \resumeItemListEnd

    \resumeSubheading
      {\href{https://www.jx2lee.kr/career/nhn-enterprise}{NHN Enterprise}}{분당구, 대한민국}
      {Cloud Operation Engineer}{Nov 2020 -- Sep 2022}
      \resumeItemListStart
        \resumeItem{Cloud Operations}
          {클라우드 검색 플랫폼 Log \& Crash Search 를 운영했습니다. CSP SLA 를 준수하기 위해 안정적인 서비스를 제공했으며, 서비스에 사용되는 데이터 오프소스들을 가까이서 만져볼 수 있는 경험을 쌓았어요}
        \resumeItem{Cloud Service Deployment}
          {NHN Cloud 서비스를 배포하고 CICD 파이프라인을 관리했습니다. 서비스 개발배포(CI/CD)의 큰 라이프사이클을 경험, 개발자들 요구사항을 만족하는 배포 모듈을 유지보수 했습니다}
      \resumeItemListEnd

    \resumeSubheading
      {\href{https://www.jx2lee.kr/career/tmax}{TmaxData}}{분당구, 대한민국}
      {Technical Support Engineer}{Aug 2018 -- Nov 2020}
      \resumeItemListStart
        \resumeItem{Technical Support}
          {고객의 기술 문제를 해결하고, 제품 사용에 대한 교육을 제공했습니다. 클라이언트 와의 커뮤니케이션을 통해 제품 피드백을 수집하고 이를 개발팀에 전달하며, 프로젝트와 PoC 를 통해 제품 개선에 기여했습니다}
      \resumeItemListEnd

  \resumeSubHeadingListEnd


%-----------PROJECTS-----------------
\section{Projects}
  \resumeSubHeadingListStart
    \resumeSubItem{\href{https://www.jx2lee.kr/opensource-contributions}{Open Source Contributions}}
      {나에게 도움을 준 오픈소스에게 보답하고자 진행한 오픈소스 기여}
    \resumeSubItem{\href{https://www.jx2lee.kr/career/projects/api-serving/}{Data API}}
      {데이터웨어하우스의 데이터를 API 로 제공하는 서비스 개발}
    \resumeSubItem{\href{https://www.jx2lee.kr/career/projects/cdc-pipeline-with-debezium}{CDC Pipelines w/ Debezium}}
      {실시간 데이터와 변경사항 추적을 목표로 개발한 CDC 파이프라인}
    \resumeSubItem{\href{https://www.jx2lee.kr/data/dbt/__/dbt-cicd-pipeline}{dbt cicd pipeline}}
      {dbt 모델 변경 감지 \& 배포까지 자동화한 GitHub Actions 기반 CI/CD 파이프라인}
    \resumeSubItem{\href{https:///www.jx2lee.kr/career/projects/self-serve-data}{Self-Serve-Data}}
      {자체 데이터 탐색 및 분석이 가능하도록, 비개발자 대상 Self-Serve 데이터 플랫폼 제공}
  \resumeSubHeadingListEnd

%
%--------SKILLS------------
\section{Skills}
\begin{tabular}{@{} l @{\hspace{2em}} l @{}}
  \textbf{Languages}                 & Python, SQL, Rust \\
  \textbf{DW/DB}                     & Redshift, BigQuery, MySQL, Oracle \\
  \textbf{Container Ochestration}     & Kubernetes, Docker, ArgoCD \& Helm(GitOps) \\
  \textbf{Public Cloud}              & AWS, GCP \\
\end{tabular}

%-------------------------------------------
\end{document}
